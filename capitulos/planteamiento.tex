\section{Planteamiento del problema}

El desarrollo de la tecnología a través de los años nos ha permitido (evolucionar) los métodos de almacenamiento y transmisión de datos en diversos ámbitos en los que se aplican las tecnologías de la información, uno de los ámbitos en los que los los avances tecnológicos han ayudado enormemente al desarrollo de el mismo es el de las ciencias meteorológicas.

Un tema recurrente en las tecnologías de la información aplicadas al ámbito de la meteorología es el estudio de diversos métodos para el almacenamiento y distribución de datos, tales como la temperatura, la presión atmosfèrica y la velocidad del viento, con un enfoque orientado a la compresión y fidelidad de los mismos para los fines que se requieran.

En una gran mayoría de las estaciones meteorológicas que son actualmente utilizadas para la recolección de datos, los métodos de recolección, tratamiento y envío de datos ha quedado definido por el equipo de hardware que se posee, y estancado en el tiempo por la falta de innovación o de recursos económicos para continuar mejorándolos.

Sin embargo, debido a la creciente necesidad de datos con una mayor precisión, así como los diversos problemas que supone la transferencia de datos desde diversos puntos con diversas infraestructuras\dots


\subsection{Antecedentes}

El ámbito del

En el año 2013 Stanchev \cite{Improved_Stanchev} propone un método de compresión de datos


Debido a diversos factores como el costo del procesamiento en dispositivos de gama baja /cite{Articulo viejo} y el costo de el almacenamiento en general \cite{Marshall_1994} se optó por el (discretizar, racionalizar) los datos climatológicos en una base de datos, estableciendo un tiempo arbitrario para la recolección de los mismos, y almacenando en las bases de datos el valor discreto de un intervalo de tiempo.

Esto es posible observarlo en uno de los equipos más comerciales para la recolección de datos meteorológicos, los dispositivos de la familia Davis, los cuales permiten establecer un intervalo de tiempo fijo en el que se realiza la recolección de datos. Este intervalo varía en diversas instituciones de 5 a 20 minutos /cite{documento de alguna institución climatologica}, y si bien es útil para el análisis de los datos en ambientes comúnes para el análisis de los datos, carecen de precisión debido a diferentes factores, sobre todo para el análisis de cambios drásticos en lapsos cortos de tiempo.

Debido al desarrollo y cambio de las tecnologías en los años recientes, se ha vuelto más económico el procesamiento y almacenamiento de datos, lo cual nos permite el establecer uevas metodologías para el procesamiento de los mismos en un ámbito común.

\subsection{Definición del problema}



En el presente documento, se hará una propuesta para aprovechar las ventajas que nos ofrecen las tecnologías actuales para crear un método de almacenamiento de datos basado en la compresión de datos

En el presente documento, se hará una propuesta para solucionar la compresión de  y equipo actual para crear un método de almacenamiento de diversos datos meteorológicos, como lo son la temperatura y la presión barométrica,
