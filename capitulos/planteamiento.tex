\section{Planteamiento del problema}

Se busca por medio de el uso de las transformadas de Fourier, el incrementar la precisión de los datos recabados de una estación climatológica, sin incrementar drásticamente el tamaño de los mismos. Aplicando técnicas de compresión al momento de generarlos.

\subsection{Antecedentes}

Debido a diversos factores como el costo del procesamiento en dispositivos de gama baja \cite{Articulo viejo} y el costo de el almacenamiento en general \cite{Libro de la biblioteca} se optó por el (discretizar, racionalizar) los datos climatológicos en una base de datos, estableciendo un tiempo arbitrario para la recolección de los mismos, y almacenando en las bases de datos el valor discreto de un intervalo de tiempo.

Esto es posible observarlo en uno de los equipos más comerciales para la recolección de datos meteorológicos, los dispositivos de la familia Davis, los cuales permiten establecer un intervalo de tiempo fijo en el que se realiza la recolección de datos. Este intervalo varía en diversas instituciones de 5 a 20 minutos \cite{documento de alguna institucion climatologica}, y si bien es útil para el análisis de los datos en ambientes comúnes para el análisis de los datos, carecen de precisión debido a diferentes factores, sobre todo para el análisis de cambios drásticos en lapsos cortos de tiempo.

Debido al desarrollo y cambio de las tecnologías en los años recientes, se ha vuelto más económico el procesamiento y almacenamiento de datos, lo cual nos permite el establecer uevas metodologías para el procesamiento de los mismos en un ámbito común.

\subsection{Definición del problema}

En el presente documento, se utilizarán las metodologías y equipo actual para crear un método de almacenamiento de datos meteorológicos (precisamente, temperatura y presión barométrica) con una precisión mayor a la que se tiene actualmente sin aumentar dramáticamente el espacio consumido por los datos, haciendo uso de métodos de compresión que tengan como base transformadas de Fourier.
