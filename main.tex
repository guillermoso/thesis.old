\documentclass{article}
\usepackage[utf8]{inputenc}
\usepackage[spanish]{babel}
\usepackage{graphicx,titling}
\usepackage[normalem]{ulem}
\usepackage{mathrsfs}
\usepackage{url}

\title{Compresión de datos meteorológicos en una estación por medio de transformadas de Fourier}
\author{Teo González Calzada}

\date{\today}

\hyphenation{uni-code}

\begin{document}

\include{capitulos/portada}
\clearpage

% \tableofcontents

\clearpage

% Fechas importantes
% Domingo 15 de marzo para entregar todo. (Sólo falta metodología y cronograma de actividades).
% Domingo 22 de marzo definitivo, para enviarlo a los evaluadores. Metodología, proupesta de solución, (marco teórico después)
% Regresando de vacaciones presentar

% En el marco teorico solo se presenta infrmación general.

%  En los antecedentes van descripciones serias de Fechas, lugares y datos historicos.

%\listoffigures
% \clearpage
%\listoftables
% \clearpage
%\maketitle

\section{Planteamiento del problema}

Diversos métodos de compresión son ampliamente utilizados en diversos ámbitos del

\subsection{Antecedentes}

Por método general, se ha decidido almacenar los datos estadísticos por medio de

\subsection{Definición del problema}

\section{Justificación}

Con alternativas más accesibles económicamente para la creación de estaciones para la medición de datos meteorológicos \cite{hernandezimplementacion} y la implementación de las mismas en diversos puntos con un acceso poco fiable a internet, se vuelve crítico el desarrollar un sistema de compresión de datos que nos permita aumentar la densidad de los datos transferidos, sin afectar la fiabilidad de los mismos.

Se pretende establecer una metodología para la compresión de datos que pueda ser implementada en diversos equipos de medición y que sea escalable, para permitir una más sencilla recolección de información en sistemas de alta densidad de datos.

\section{Marco Teórico}

\begin{figure}
   \centering
   \includegraphics[scale=0.5]{images/escudo-uacj.png}
   \label{fig:escudero}
   \caption[esc]{escudero}

\end{figure}

Transformadas de Fourier
Meteorología
   Presión barométrica
   Temperatura

Normalizacion de Temperatura
NOAA

$\mathscr{L}\{f(t)\}=F(s)$


\section{Objetivo general}

Diseñar e implementar un método para la compresión de los datos meteorológicos basado en transformadas de ondícula que ofrezca una eficiencia mayor al respecto del tamaño de los datos que el actual utilizado.

\section{Objetivos específicos}

\begin{itemize}
   \item Diseñar e implementar filtros para normalización de datos, adecuados para las variables climatológicas.
   \item Diseñar e implementar un método de compresión de datos basado en transformadas de ondícula.
   \item Implementar el método de compresión en Python.
   \item Probar el rendimiento del método de compresión con datos generados.
   \item Integrar de el método de compresión propuesto con una estación meteorológica utilizando el software Weewx para la recolección de datos.
   \item Diseñar un sitio web para la consulta y visualización de los datos generados por la estación meteorológica.
\end{itemize}

\section{Metas (Solución propuesta)}

Se creará un sistema de compresión de datos para muestras a corto plazo de estaciones climatológicas. En el cual se

\section{Alcances y limitaciones}

El programa tendrá como fin el integrarse solamente con el software de recolección meteorológica abierto 'weewx', para almacenar los datos comprimidos en un sistema de base de datos no relacional (NoSQL),

\section{Cronograma de Actividades}


Actividades.

Investigación e implementación de filtros para normalización de datos.


Investigación de métodos de compresión




\bibliographystyle{ieeetr}
\bibliography{biblio}

\end{document}
